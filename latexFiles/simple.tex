\begin{figure}[!htb]
\centering
\subfigure{\centering \includegraphics[width=0.47\textwidth]{figures/results/simple_features/color_corrs.png}}
\vspace{-5mm}\caption{\footnotesize\textbf{Simple color features do not explain object memorability.} Correlations of object memorability scores with hue and saturation are near zero, and only value shows a very weak correlation.}\label{fig:simple}
\end{figure}

While simple low-level image features are traditionally poor
predictors of image memorability \cite{isola11} (with good reason
\cite{konkle10}), the question arises whether such features play any
role in determining object memorability in images. To address this
query and following a similar strategy as in \cite{isola11,isola14},
we compute the mean (and variance) of HSV color of each object in our
dataset, i.e. first and second order  statistics of pixel color within
an object's ground truth segmentation, and correlate it (Spearman rank
correlation) with the underlying object memorability score (refer to
Figure \ref{fig:simple}). We see that the mean ($\rho = 0.1$) and
variance ($\rho=0.25$) of the V channel show weak correlation with
object memorability, suggesting that brighter and higher contrast
objects may be more memorable. On the other hand, essentially no
relationship exists between memorability and either the H or S
channels. This slightly deviates from the findings in \cite{isola11},
which show mean hue to be weakly predictive of image
memorability. This difference could be due to the fact that many
images of the dataset in \cite{isola11} show blue and green outdoor
landscapes as being less memorable than warmly colored human faces and
indoor scenes, while our dataset contains plenty of indoor objects and
people and outdoor scene-related segments such as sky and ground are
not included as objects. From these results, we see that, like image
memorability, simple pixel statistics do not play a significant role
in determining object memorability in images.

