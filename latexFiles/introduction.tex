\begin{figure}[t]
\centering
\subfigure{\centering \includegraphics[width=0.2\textwidth]{figures/introduction/113.png}}
\subfigure{\centering \includegraphics[width=0.2\textwidth]{figures/introduction/intro.png}}
\vspace{-5mm}\caption{\footnotesize\textbf{Not all objects are equally remembered.} Memorability scores of objects for the image in the top row obtained from our pysophysics experiment. }\label{fig:introPhoto}
\end{figure}

Consider the image and its corresponding objects in Figure \ref{fig:introPhoto}. Even though the person on the right is comparable in size to person on the left, he is remembered far less by human subjects (indicated by their respective memorability scores of $0.18$ and $0.64$). Moreover, people tend to remember the person on the left and the fish in the center, even after $3$ minutes and more than 70 additional stimuli have intervened (memorability score $= 0.64$). Interestingly, despite vibrant colors and considerable size, the boat is far less memorable (score $= 0.18$).

Considerable effort has recently been aimed at understanding and predicting image memorability, and with good reason: such a property of images is useful to understand as it can be leveraged in a variety of fascinating applications [cite papers here including modifying face memorability]. However, such an analysis does not provide direct information about just what exactly in those images are the memorable parts. Consider again the example in Figure \ref{fig:introPhoto}. What is it about the fish and the person on the left that makes them much more memorable than the other objects in the image? Moreover, how does this object level memorability influence the overall memorability of the photo? Is it by virtue of a single object or a combination/configuration of multiple objects? Inferring such detailed information from image-level knowledge alone is an extremely difficult task.. An understanding of memorability at an object level provides relevant information at a more granular level and may also eventually produce bottom up explanation of image memorability. Such a complete picture about memorability can provide vision systems with information that humans consider meaningful which can lead to a wide variety of applications. As an analogy to what humans remember and consider meaningful in an image, we draw attention to the large body of work on saliency that has sought to determine what is “important” in an image.Saliency, like memorability and aesthetic value, is an image property that machines must leverage in order to be utilized effectively by humans. What is more, these properties may influence each other as well. For example, while memorability does not appear to be increased by aesthetic value, saliency may actively interact with it (cite zoya and icip paper here). To what extent do gaze patterns correlate with memorability, and more generally, to what extent are visual saliency and memorability related?

In this paper, we systematically explore the memorability of objects within individual images and shed light on the various factors that drive object memorability. In exploring the connection between object memorability, saliency, and image memorability, our paper makes several important contributions.

Firstly, we show that just like image memorability, object memorability is a property that is shared across subjects: objects remembered by one person are also likely to be remembered by others and vice versa. Secondly, we uncover the relationship between visual saliency and object memorability, and demonstrate those instances where visual saliency directly predicts object memorability and when/why it fails. While there have been a few studies that explore the connection between image memorability and visual saliency \cite{zoya15}, \cite{lemeur13}, our work is the first to explore the connection between object memorability and visual saliency. Third, we make the initial leaps in disambiguating the link between image memorability and object memorability, and show that in many cases, the memorability of an image is primarily driven by the memorability of it’s most memorable object. Studying these questions, help not only understand visual saliency, image and object memorability in more detail, but it can also have important contributions to computer vision. For example, understanding which regions and objects in an image are memorable would enable us to modify the memorability of images which can have applications in advertising, user interface design etc. With this in mind, we show in section 4 that our proposed dataset can serve as a benchmark for evaluating object memorability algorithms and encourage future object and region memorability prediction schemes. Taken together, our efforts offer a deeper understanding of memorability in general and broaden the contribution of human-level insight to improve machine prediction

\subsection{Related works}
\input{relatedWork} 