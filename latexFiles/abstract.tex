Recent studies on image memorability have shed light on what distinguishes the memorability of different images and the intrinsic and extrinsic properties that make those images memorable. However, a clear understanding of the memorability of specific objects inside an image remains elusive. In this paper, we provide the first attempt to answer the question: what exactly about an image is remembered? We augment both the images and object segmentations from the PASCAL-S dataset with ground truth memorability scores and shed light on the various factors and properties that make an object memorable (or forgettable) to humans. We analyze various visual factors that may influence object memorability (e.g. color, visual saliency, and object categories). We also study the correlation between object and image memorability and find that image memorability is greatly affected by the memorability of its most memorable object. Lastly, we explore the effectiveness of deep learning and other computational approaches in predicting object memorability in images. Our efforts offer a deeper understanding of memorability in general thereby opening up avenues for a wide variety of applications.

