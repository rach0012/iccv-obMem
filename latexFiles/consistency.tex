%Previous work on image memorability has found human consistency to be fairly high. That is, people tend to remember the same images, and exhibit similar performance in doing so. Despite variability due to individual differences and other sources of noise, this level of consistency provides evidence that memorability is an intrinsic property of images that can be predicted. In contrast to full images, this paper focuses primarily on the memorability of individual objects in an image, which may or may not exhibit the same level of human consistency as full images, which often contain complex arrangements of several objects. High consistency in object memorability would indicate that, like full images, objects can potentially be predicted with high accuracy.
To assess human consistency in remembering objects, we repeatedly divided our entire subject pool into two equal halves and quantified the degree to which memorability scores for the two sets of subjects were in agreement using Spearman’s rank correlation ($\rho$). We computed the average correlation over $25$ of these random split iterations, yielding an average correlation of $\rho=0.76$. This high consistency in object memorability indicates that, like full images, object memorability is a shared property across subjects. People tend to remember (and forget) the same objects in images, and exhibit similar performance in doing so. Thus memorability of objects in images can potentially be predicted with high accuracy. In the next section, we study the various factors that possibly drive object memorability in images.

%this level of
%consistency suggests that information intrinsic to the images
%might be used by different people to remember them.
%In the next section, we search for this image information.
