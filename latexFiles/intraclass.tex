As demonstrated above, some object classes (i.e. animal, person, vehicle) are more memorable than others. However, not all objects in a class are equally memorable. The examples in figure \ref{fig:qualLabel} show the most memorable and least memorable objects for each object class. Across classes,  non-memorable objects tend to be those that are occluded and obstructed by other objects. What other possible factors could influence the memorability of an object within a class? Among the various possible factors, we explored how object memorability within a particular class is influenced by a) the number of objects in an image and b) the presence of other object classes.

\textbf{Number of objects:} We first examined how the memorability of each object class is affected by the number of objects inside an image. Figure \ref{fig:obLabelChange} shows the change in average memorability of the different object classes with respect to the increase in number of objects in an image. Results indicate that the number of objects present in an image is an important factor in determining memorability. For example, as the number of objects in an image increases, the memorability  animals and vehicles decreases sharply, likely as a result of competition for attention, or decreased spotlight on a single subject of the composition.  Interestingly, the memorability of the person class does not change significantly with an increase in number of objects. This suggests that people are not only one of the most memorable object classes, but are also more robust to the presence of clutter in images. This may be because single people in images steal all of the attention of the viewer, but how do they behave in the presence of other people? To answer this, we turn to the question of interclass memorability next.

\begin{figure}[t]
\centering
\subfigure{\centering \includegraphics[width=0.45\textwidth]{figures/results/obLabel/memScore_change2.png}}
\vspace{-5mm}\caption{\footnotesize\textbf{Correlation between object class and number of objects.} add-in later. }\label{fig:obLabelChange}
\end{figure}

\textbf{Inter-class memorability:} How does the presence of a particular object class influence the memorability of another object class? For all pairwise combinations of object category, we gathered all images that contained at least one object from both categories and computed the change in the average memorability scores for the two object categories. Figure \ref{fig:obLabelPair} plots these data and  visualizes how the memorability of each object class is affected in the presence of all other object classes. The first thing to note is that the values of low-memorability classes (i.e. nature, furniture, device, and building) are not greatly affected by the presence of other object categories. Instead, their memorability tends to remain low in all contexts. The memorability of the animal class remains close to its high average memorability score in presence of most classes, but drops significantly in the presence of vehicles and people. The memorability of people also remains close to its average memorability score and tends to be unaffected by the presence of most object categories, as found previously. However, the memorability of people  decreases  in the presence of vehicles and buildings. This could be due to the fact that people in images containing vehicles or buildings are usually zoomed out and are usually smaller in size (also illustrated in figure \ref{fig:qualInterClass}). The memorability of the vehicle class is strongly affected by the presence of other object categories. In particular, its memorability drops significantly in the presence of people and animals. Taken together, when an animal, vehicle or a person is present in the same image, the memorability of all three classes usually goes down. However, this pattern of change in memorability varies by class, leading to interesting results. For example, when a vehicle or an animal co-occurs with a person, the person is generally more memorable. When a vehicle and animal are present in the same image, the animal is generally more memorable. even though the memorability of both of these classes drops significantly. Refer to Figure \ref{fig:qualInterClass} for further examples of  inter-class memorability effects.

\begin{figure}[b]
\centering
\subfigure{\centering \includegraphics[width=0.5\textwidth]{figures/results/obLabel/confusionMatrix.png}}
\vspace{-5mm}\caption{\footnotesize\textbf{inter-class object memorability relationship.} add-in later. }\label{fig:obLabelPair}
\end{figure}

\begin{figure}[t]
\centering
\subfigure{\centering \includegraphics[height = 1.1 cm]{figures/results/obLabel/inter-class/14.jpg}}
\subfigure{\centering \includegraphics[height = 1.1 cm]{figures/results/obLabel/inter-class/85.jpg}}
\subfigure{\centering \includegraphics[height = 1.1 cm]{figures/results/obLabel/inter-class/10.jpg}}
\subfigure{\centering \includegraphics[height = 1.1 cm]{figures/results/obLabel/inter-class/128.jpg}}
\subfigure{\centering \includegraphics[height = 1.1 cm]{figures/results/obLabel/inter-class/102.jpg}}
\subfigure{\centering \includegraphics[height = 1.1 cm]{figures/results/obLabel/inter-class/129.jpg}}\\
\subfigure{\centering \includegraphics[height = 1.1 cm]{figures/results/obLabel/inter-class/14.png}}
\subfigure{\centering \includegraphics[height = 1.1 cm]{figures/results/obLabel/inter-class/85.png}}
\subfigure{\centering \includegraphics[height = 1.1 cm]{figures/results/obLabel/inter-class/10.png}}
\subfigure{\centering \includegraphics[height = 1.1 cm]{figures/results/obLabel/inter-class/128.png}}
\subfigure{\centering \includegraphics[height = 1.1 cm]{figures/results/obLabel/inter-class/102.png}}
\subfigure{\centering \includegraphics[height = 1.1 cm]{figures/results/obLabel/inter-class/129.png}}
\vspace{-5mm}\caption{\footnotesize\textbf{Qual Results.} Figure showing how memorability of different classes is effected in presence of other classes. Bottom row is the memorabilty map}\label{fig:qualInterClass}
\end{figure}

