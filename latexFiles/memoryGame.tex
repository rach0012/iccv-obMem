To measure the memorability of individual objects in each dataset image \B{use either database or dataset consistently throughout the paper}, we created an alternate version of the Visual Memory Game through Amazon Mechanical Turk following the basic design in \cite{isola11}, with the exception of a few key differences (refer to Figure \ref{fig:mainTask}). In our game, participants first viewed a sequence of images one at a time \B{how many images in this sequence?}, with a $1.5$ second interval between image presentations. The subjects were asked to remember the contents and objects inside these images as much as they could. To ensure that subjects would not only just  look at the salient or center objects, they were given unlimited time to freely view the images. Once they were done viewing an image, they could press any key to advance to the next image. After the initial image sequence, participants viewed a sequence of objects \B{how many objects in this sequence?}, their task then being to indicate through a key press which of those objects was present in one of the previously shown images. Each object was displayed for $1.5$ second, with a $1.5$ second gap in between object presentations. Pairs of corresponding image and object sequences were broken up into $10$ blocks. Each block consisted of $80$ total stimuli ($35$ images and $45$ objects), and lasted approximately $3$ minutes. At the end of each block, the subject could take a short break. Overall, the experiment takes approximately $30$ minutes to complete.

Unknown to the subjects, each sequence of images inside each block was pseudo-randomly generated to consist of $3$ `target' images taken from the Pascal-S dataset, whose objects were later presented to the participants for identification. The remaining images in the sequence consisted of $16$ `filler' images and $16$ `familiar' images. `Filler' images were randomly selected from the DUT-OMRON dataset \cite{dutomron13}, while the `familiar' ones were randomly sampled from the MSRA dataset \cite{msra11}. In a similar fashion, the object sequence in each block was also generated pseudo-randomly to consist of $3$ `target' objects ($1$ object taken randomly from each previously shown target image). The remaining objects in the sequence consisted of $10$ `control', $16$ `filler', and $16$ `familiar' objects. `Filler' objects were sampled randomly from the $80$ different object categories in the Microsoft COCO dataset \cite{coco14}, while the `familiar' objects were sampled from objects taken from the previously displayed `familiar' images in the image sequence. The fillers and familiars helped provide spacing between the target images and target objects, whereas the control objects allowed us to check if the subjects were paying attention to the task \cite{brady08,isola11}. While the fillers and familiars (both images and objects) were taken from datasets of real world scenes and objects, the `control' objects were artificial stimuli randomly sampled from the dataset proposed in \cite{brady08} and served as a control to test subject attentiveness. Target images and their corresponding target objects were spaced $70-79$ stimuli apart, while familiar images and their objects were spaced $1-79$ stimuli apart \B{is 1 here a typo? is there logic behind this difference?}. All images and objects appeared only once, and each subject was tested on only one object from each target image. Objects were centered within the image they originated from and non-object pixels were set to grey. Participants were required to complete the entire task, which included $10$ blocks ($\sim$$30$ minutes) and could not participate in the experiment a second time. After collecting the data, we assign a `memorability score' to each target object in our dataset, defined as the percentage of correct detections by subjects (refer to Figure \ref{fig:introPhoto} for an example). In all our analysis, we remove all subjects whose accuracy on the control objects is below $70\%$. In the end, this memorability game had a total of $1823$ workers from Mechanical Turk with more than $95\%$ approval rating in Amazon’s system. The memorability score of an object corresponds to the number of subjects that correctly detected the presence of that object in a previously seen image. On average, each object was scored by $16$ subjects and the average memorability score was $33\%$ with a standard deviation of  $28\%$ \B{Is this a large deviation? does it need explanation?}.
