\begin{figure*}[!thb]
\centering
\subfigure{\centering \includegraphics[width=1\textwidth]{figures/results/imMem/qual.png}}
\vspace{-16pt}\caption{\footnotesize\textbf{Max object memorability predicts image memorability.} Top row: most memorable images taken from our dataset along with their highest memorable object and their respective memorability scores. Bottom row: least memorable images in the dataset along with their most memorable object and their respective memorability scores. We notice that the maximum object memorability correlates strongly with image memorability in both the cases. }\label{fig:imMemQual}%\vspace{-12pt}
\end{figure*}

Until now, we have studied what objects people remember and the factors that influence their memorability, but to what extent does the memorability of individual objects affect the overall memorability of an image? Moreover, if an image is highly memorable, what can we say about the memorability of the objects inside those images (and vice versa)? To shed light on these queries, we conducted a second large-scale experiment on Amazon Mechanical Turk for all images in our dataset to gather their respective \emph{image} memorability scores. For this experiment, we followed the same strategy as the memory game experiment proposed in \cite{isola11}. A series of images from our dataset and Microsoft COCO dataset \cite{coco14} (i.e. `filler' images) were flashed for $1$ second each, and subjects were instructed to press a key whenever they detected a repeat presentation of an image. A total of $350$ workers participated in this experiment with each image being viewed $80$ times on average. The rank correlation after averaging over $25$ random splits was $0.7$, thus, validating annotator consistency in  the image memorability scores.






Using results from the previous experiments, we computed the correlation between the scores of the single most memorable \emph{object} in each image and the memorability score of each \emph{image}. This correlation is moderately high with $\rho=0.4$, suggesting that the most memorable object in an  image plays a crucial role in determining the overall memorability of an image. To investigate this finding in relation to some extreme cases, we repeated the same analysis as above but on a subset of the data containing the $100$ most memorable images and the $100$ least memorable images. The correlation between maximum object memorability and image memorability for this subset of images increased significantly to $\rho=0.62$. This means that maximum object memorability serves as a strong indicator of whether an image is \textit{highly} memorable or \emph{not} memorable at all. In other words, images that are highly memorable contain at least one highly memorable object and images with low memorability usually do not contain a single highly memorable object (refer to Figure \ref{fig:imMemQual}).

It seems that maximum object memorability is highly explanatory, but does this behavior generalize across object categories? We further computed the correlation between maximum object and image memorability for each individual object category. The correlation for the categories were: animal ($\rho=0.38$), building ($\rho=0.22$), device ($\rho=0.47$), furniture ($\rho=0.53$), nature ($\rho=0.64$), person ($\rho=0.54$), and vehicle ($\rho=0.30$) which shows that certain categories are more strongly correlated than others. For example, images containing animals, buildings, or vehicles as the most memorable objects tend to have varying degree of image memorability (indicated by their lower $\rho$ values). On the other hand, device, furniture, nature, and person are strongly correlated with image memorability, indicating that if an image's most memorable object belongs to one of these categories, the object memorability score is strongly predictive of the image memorability score. We can imagine scenarios in which this information is potentially useful. For example, in vision systems that are tasked to predict scene memorability, a \textit{single} object and its category can serve as a strong prior in predicting this score.






