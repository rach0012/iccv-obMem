As a first step towards understanding memorability of objects in
images, we compile an image database containing a variety of objects
from a diverse range of categories, as well as, measure the
probability that every object in each database image will be
remembered by a large group of subjects after a single viewing.
%
This
helps provide ground truth memorability scores for objects inside
images (defined as image segments) and allows for a precise analysis
of the memorable elements within an image.


%
Toward this, we utilized the
PASCAL-S dataset \cite{yin14}, a fully segmented dataset built on the
validation set of the PASCAL VOC 2010 \cite{pascal10} segmentation
challenge. To improve segmentation quality, we manually refined the
segmentations from this dataset. We removed all homogenous non-object
or background segments (e.g. ground, grass, floor, and sky), along
with imperceptible object fragments and excessively blurred
regions. All remaining object segmentations were tested for
memorability. In summary, our final dataset comprises $850$ images and
$3,412$ object segmentations (i.e. an average of $4$ objects per
image), for which we gathered ground truth memorability through crowd
sourcing. %\B{how about image size? - they were all in their original
          %sizes and aspect ratios.}


\subsection{Object Memory Game}
To measure the memorability of individual objects in each dataset image %\B{use either database or dataset consistently throughout the paper}
, we created an alternate version of the Visual Memory Game through Amazon Mechanical Turk following the basic design in \cite{isola11}, with the exception of a few key differences (refer to Figure \ref{fig:mainTask}). In our game, participants first viewed a sequence of $35$ images one at a time, with a $1.5$ second interval between image presentations. The subjects were asked to remember the contents and objects inside these images as much as they could. To ensure that subjects would not only just  look at the salient or center objects, they were given unlimited time to freely view the images. Once they were done viewing an image, they could press any key to advance to the next image. After the initial image sequence, participants viewed a sequence of $45$ objects, their task then being to indicate through a key press which of those objects was present in one of the previously shown images. Each object was displayed for $1.5$ seconds, with a $1.5$ second gap in between each object in the sequence. Pairs of corresponding image and object sequences were broken up into $10$ blocks. Each block consisted of $80$ total stimuli ($35$ images and $45$ objects), and lasted approximately $3$ minutes. At the end of each block, the subject could take a short break. Overall, the experiment takes approximately $30$ minutes to complete.

Unknown to the subjects, each sequence of images inside each block was pseudo-randomly generated to consist of $3$ ``target'' images taken from the Pascal-S dataset, whose objects were later presented to the participants for identification. The remaining images in the sequence consisted of $16$ ``filler'' images and $16$ ``familiar'' images. 
Filler images were randomly selected from the DUT-OMRON dataset \cite{dutomron13}, while the familiar ones were randomly sampled from the MSRA dataset \cite{msra11}. In a similar fashion, the object sequence in each block was also generated pseudo-randomly to consist of $3$ target objects ($1$ object taken randomly from each previously shown target image). The remaining objects in the sequence consisted of $10$ control, $16$ filler, and $16$ familiar objects. Filler objects were sampled randomly from the $80$ different object categories in the Microsoft COCO dataset \cite{coco14}, while the familiar objects were sampled from objects taken from the previously displayed familiar images in the image sequence. The fillers and familiars helped provide spacing between the target images and target objects, whereas the control objects allowed us to check if the subjects were paying attention to the task \cite{brady08,isola11}. While the fillers and familiars (both images and objects) were taken from datasets of real world scenes and objects, the control objects were artificial stimuli randomly sampled from the dataset proposed in \cite{brady08} and served as a control to test subject attentiveness. Target images and their corresponding target objects were spaced $70-79$ stimuli apart, while familiar images and their objects were spaced $1-79$ stimuli apart. All images and objects appeared only once, and each subject was tested on only one object from each target image. Objects were centered within the image they originated from and non-object pixels were set to grey. Participants were required to complete the entire task, which included $10$ blocks ($\sim$$30$ minutes) and could not participate in the experiment a second time. After collecting the data, we assign a memorability score to each target object in our dataset, defined as the percentage of correct detections by subjects (refer to Figure \ref{fig:introPhoto} for an example). In all our analysis, we remove all subjects whose accuracy on the control objects is below $70\%$. In the end, this memorability game had a total of $1,823$ workers from Mechanical Turk with more than $95\%$ approval rating in Amazon’s system. The memorability score of an object corresponds to the number of subjects that correctly detected the presence of that object in a previously seen image. On average, each object was scored by $16$ subjects and the average memorability score was $33\%$ with a standard deviation of  $28\%$. %\B{Is this a large deviation? does it need explanation? no I don't think so.. some objects were barely memorable and some of them had a high memorability}.   


\subsection{Consistency Analysis}
%Previous work on image memorability has found human consistency to be fairly high. That is, people tend to remember the same images, and exhibit similar performance in doing so. Despite variability due to individual differences and other sources of noise, this level of consistency provides evidence that memorability is an intrinsic property of images that can be predicted. In contrast to full images, this paper focuses primarily on the memorability of individual objects in an image, which may or may not exhibit the same level of human consistency as full images, which often contain complex arrangements of several objects. High consistency in object memorability would indicate that, like full images, objects can potentially be predicted with high accuracy.
To assess human consistency in remembering objects, we repeatedly divided our entire subject pool into two equal halves and quantified the degree to which memorability scores for the two sets of subjects were in agreement using Spearman’s rank correlation ($\rho$). We computed the average correlation over $25$ of these random split iterations, yielding an average correlation of $\rho=0.76$. This high consistency in object memorability indicates that, like full images, object memorability is a shared property across subjects. People tend to remember (and forget) the same objects in images, and exhibit similar performance in doing so. Thus memorability of objects in images can potentially be predicted with high accuracy. In the next section, we study the various factors that possibly drive object memorability in images.

%this level of
%consistency suggests that information intrinsic to the images
%might be used by different people to remember them.
%In the next section, we search for this image information.

